%%%%%%%%%%%%%%%%%%%%%%%%%%%%%%%%%%%%%%%%%
% Beamer Presentation
% LaTeX Template
% Version 1.0 (10/11/12)
%
% This template has been downloaded from:
% http://www.LaTeXTemplates.com
%
% License:
% CC BY-NC-SA 3.0 (http://creativecommons.org/licenses/by-nc-sa/3.0/)
%
%%%%%%%%%%%%%%%%%%%%%%%%%%%%%%%%%%%%%%%%%

%----------------------------------------------------------------------------------------
%	PACKAGES AND THEMES
%----------------------------------------------------------------------------------------

\documentclass{beamer}

\mode<presentation> {

% The Beamer class comes with a number of default slide themes
% which change the colors and layouts of slides. Below this is a list
% of all the themes, uncomment each in turn to see what they look like.

%\usetheme{default}
%\usetheme{AnnArbor}
%\usetheme{Antibes}
%\usetheme{Bergen}
%\usetheme{Berkeley}
%\usetheme{Berlin}
%\usetheme{Boadilla}
%\usetheme{CambridgeUS}
%\usetheme{Copenhagen}
%\usetheme{Darmstadt}
%\usetheme{Dresden}
%\usetheme{Frankfurt}
%\usetheme{Goettingen}
%\usetheme{Hannover}
%\usetheme{Ilmenau}
%\usetheme{JuanLesPins}
%\usetheme{Luebeck}
\usetheme{Madrid}
%\usetheme{Malmoe}
%\usetheme{Marburg}
%\usetheme{Montpellier}
%\usetheme{PaloAlto}
%\usetheme{Pittsburgh}
%\usetheme{Rochester}
%\usetheme{Singapore}
%\usetheme{Szeged}
%\usetheme{Warsaw}

% As well as themes, the Beamer class has a number of color themes
% for any slide theme. Uncomment each of these in turn to see how it
% changes the colors of your current slide theme.

%\usecolortheme{albatross}
%\usecolortheme{beaver}
%\usecolortheme{beetle}
%\usecolortheme{crane}
%\usecolortheme{dolphin}
%\usecolortheme{dove}
%\usecolortheme{fly}
%\usecolortheme{lily}
%\usecolortheme{orchid}
%\usecolortheme{rose}
%\usecolortheme{seagull}
%\usecolortheme{seahorse}
%\usecolortheme{whale}
%\usecolortheme{wolverine}

%\setbeamertemplate{footline} % To remove the footer line in all slides uncomment this line
%\setbeamertemplate{footline}[page number] % To replace the footer line in all slides with a simple slide count uncomment this line

%\setbeamertemplate{navigation symbols}{} % To remove the navigation symbols from the bottom of all slides uncomment this line
}

\usepackage{graphicx} % Allows including images
\usepackage{booktabs} % Allows the use of \toprule, \midrule and \bottomrule in tables

%----------------------------------------------------------------------------------------
%	TITLE PAGE
%----------------------------------------------------------------------------------------

\title[HPC/N-Body]{Course in HPC / Optimizing N-Body problem} % The short title appears at the bottom of every slide, the full title is only on the title page

\author{Guido Giuntoli} % Your name
\institute[Huawei] % Your institution as it will appear on the bottom of every slide, may be shorthand to save space
{
Software Solution Architect for Advanced Computing at Huawei\\ % Your institution for the title page
\medskip
\textit{gagiuntoli@gmail.com} % Your email address
}
\date{\today} % Date, can be changed to a custom date

\begin{document}

\begin{frame}
\titlepage % Print the title page as the first slide
\end{frame}

%----------------------------------------------------------------------------------------
%	PRESENTATION SLIDES
%----------------------------------------------------------------------------------------

\begin{frame}
\frametitle{About me}
\begin{itemize}
\item Ph.D. in HPC at Barcelona Supercomputing Center: \\
 - Develop scientific applications in C/C++ \\
 - Accelerate code with MPI/OpenMP/OpenACC/CUDA \\
 - Run and profile code in multi-node hybrid CPU+GPU architectures

\vspace{2em}
\item Working as Software Solution Architect at Huawei \\
 - Evaluate and compare performance of Kunpeng ARM Processors \\
 - Measure performance and indentify bottlenecs of applications \\
 - Enable SVE vector instructions in applications \\
\end{itemize}
\end{frame}

%------------------------------------------------

\begin{frame}
\frametitle{N-Body}

Force between particle:
$$\overline{F}_{ij} = G \frac{m_i m_j}{|r|^2} \frac{\overline{r}}{|r|}$$
Force over one particle:
$$\overline{F}_{i} = \sum_j \overline{F}_{ij}$$
Acceleration:
$$\overline{a} = \frac{\overline{F}_{i}}{m_i}$$
Velocity
$$\overline{v}^{n+1} = \overline{v}^n + \overline{a} \; \Delta t$$
Position
$$\overline{x}^{n+1} = \overline{x}^n + \overline{v}^{n+1} \; \Delta t$$

\end{frame}

%------------------------------------------------

\end{document} 
